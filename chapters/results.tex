\chapter{Eredmények}
\label{ch:results}

\section{Vizsgált kódbázisok}

Egy kódbázisnak a következő feltételeket kell teljesíteni, hogy vizsgálni tudjuk:
\begin{enumerate}
    \item Legyen könnyen build-elhető, évekre visszamenően
    \item A tesztek legyenek könnyen futtathatóak
    \item Generálódjon coverage report a teszt futtatásokhoz és a report formátuma legyen egyike azoknak, amiket támogat a ReportGenerator\footnote{\url{https://github.com/danielpalme/ReportGenerator}} nevű .NET-es könyvtár
    \item A projekt legyen nyílt forráskódú
    \item Rendelkezzen a projekt viszonylag gazdag commit történettel -- jelen esetben az 5000+ commit és 3+ éves történet heurisztikát alkalmaztam
\end{enumerate}

Ezeket a feltételeket ugyan könnyű teljesíteni egy fejlesztő csapatnak ipari környezetben saját projektekre, sajnos nyílt forráskódú környezetben más a helyzet -- specifikusan az 1-es és 3-as pontok jelentenek problémát. A C++-ra, Python-ra és .NET Framework-re épülő projektek például egy az egyben kiesnek, mert egyik sem ad egyszerű módot arra, hogy a build-eléshez szükséges környezetet könnyen elő tudjuk állítani a projekt életciklusának akármely pontján.

Gyakorlatban a JavaScript NodeJS/NPM, .NET Core/.NET 5+ és Java ökoszisztémákra épülő projektek felelhetnek meg a korábbi feltételeknek. Ezek közül azonban a Java a szakmai hátterem miatt kiesik, a .NET Core/5+ projektek pedig az 5-ös feltételt nem teljesítik, így marad a JavaScript Node/NPM ökoszisztéma. A specifikus projekteknél külön meg fogom említeni, de általánosságban itt is leírom: még a JavaScript/NodeJS/NPM ökoszisztéma esetén is komoly problémát jelent a régi (~2015 előtti) verziók build-elése, mivel az NPM és a node, illetve a rájuk épített projektek sok változáson estek át az évek során, amit ma a projekt ismerete nélkül már nehéz reprodukálni. Ez azt jelenti, hogy a projektek evolúcióját nem lehet a kezdetektől indulva mérni, csak attól a ponttól kezdve, ahol már lehetséges egy sikeres build és teszt futtatás.

Fontos leszögezni még egy dolgot. A JavaScript-es projektek elemzése egy jelentős vakfoltot fog képezni, méghozzá azért, mert a nyílt forráskódú projektek között gyakorlatilag lehetetlen olyat találni, amire igazak a fenti feltételek és az is, hogy alacsony coverage-el rendelkeznek, azaz nehezen karbantarthatóak. Több száz repository-t futottam, hogy olyat találjak, ami minőség vagy karbantarthatóság szempontjából első ránézésre "rossznak" tűnik, de a GitHub-on host-olt, a feltételeknek megfelelő projektek túlnyomó többsége 95\% feletti coverage-el rendelkezik. Ráadásul azok a projektek, amik 95\% alatt vannak, azok általában csak konzervatív coverage ignore path-ok, vagy archív, tesztek által már nem meghajtott, de nem éles kód miatt mutatnak ilyen értékeket.

A fentieket figyelembe véve a következő projektekre esett a választás:
\begin{itemize}
    \item Vue: \url{https://github.com/vuejs/vue}
    \item Moment: \url{https://github.com/moment/moment}
    \item React: \url{https://github.com/facebook/react}
\end{itemize}

Az analízishez felhasznált nyers adatok, illetve az azokból készített excel kimutatások elérhetőek GitHub-on: \url{https://github.com/marczinusd/hestia-thesis/tree/master/raw-data}

\pagebreak
\input{chapters/vue.tex}

\pagebreak
\section{Moment.js}

A második mélyebben megvizsgált kódbázis a moment.js\footnote{https://momentjs.com/} lesz. A moment éveken át volt a de facto JavaScript-ben írt dátum-idő könyvtár, azonban 2020-ban különböző okoknál fogva a fejlesztése befejeződött.

A moment kódbázisa két szempontból lesz érdekes: egyrészt "késznek" tekinthető, másrészt egy nagyon jól definiált, kis problémát volt hivatott megoldani a kezdetekből.

Ideális esetben a legelső kiadástól lenne érdemes kezdeni az analízist, azonban a moment 1.0 megjelenésekor sem unit tesztek, sem coverage report nem voltak a projekthez, ráadásul a kódbázis egy masszív JavaScript fájl volt. A moment 2.0 megjelenése azonban viszonylag közel van az 1.0-hoz és a 2.10.5-ös minor release-től kezdve elérhető a coverage report, úgyhogy az lesz a kezdőpont

A \ref{fig:moment-2.10.5-changes} ábrán látható az összes fájl módosítási száma és a fájlokhoz tartozó egyedi szerzők száma. Egy érdekes dolog már most megfigyelhető: ellentétben a vue-val, itt a szórás a különböző fájlok módosítási számai között viszonylag alacsony. Ehhez viszont hozzátartozik a korábban említett 1.0-ás kiadás, ahol a kódbázis egy \code{moment.js} nevű fájl volt, aminek a története egy újraírás miatt nem jelenik meg a 2.x-es kiadásokban. 

\begin{figure}[H]
    \centering
    \includegraphics[width=1\textwidth]{images/moment/moment-2.10.5-changes.png}
    \caption{Moment.js 2.10.5-ös kiadásában lévő fájlok módosítási számai}
    \label{fig:moment-2.10.5-changes}
\end{figure}

A \ref{fig:moment-2.10.5-hist} ábrán látható hisztogram jól demonstrálja a kódbázis állapotát a 2.10.5-ös kiadásban. Ugyan technikailag itt is látható az a trend, hogy a fájlok túlnyomó többsége módosítások számát tekintve az alsó 40\%-ban van, azonban itt ez csalóka, hiszen az módosítások számának intervalluma csak 1-től 9-ig terjed.

\begin{figure}[H]
    \centering
    \includegraphics[width=1\textwidth]{images/moment/moment-2.10.5-hist.png}
    \caption{Moment.js 2.10.5-ös kiadásában lévő fájlok módosítási számainak hisztogramja}
    \label{fig:moment-2.10.5-hist}
\end{figure}

Ugorjunk az időben a 2.20.5-ös verzióhoz tartozó snapshot-ra. A \ref{fig:moment-2.20.5-changes} ábra már sokkal jobban hasonlít a vue-nál látottakhoz: változtatások számának tekintetében 

\begin{figure}[H]
    \centering
    \includegraphics[width=1\textwidth]{images/moment/moment-2.20.5-changes.png}
    \caption{Moment}
    \label{fig:moment-2.20.5-changes}
\end{figure}

\begin{figure}[H]
    \centering
    \includegraphics[width=1\textwidth]{images/moment/moment-2.20.5-auth.png}
    \caption{Moment}
    \label{fig:moment-2.20.5-auth}
\end{figure}

\begin{figure}[H]
    \centering
    \includegraphics[width=1\textwidth]{images/moment/moment-2.20.5-hist.png}
    \caption{Moment}
    \label{fig:moment-2.20.5-hist}
\end{figure}


\begin{figure}[H]
    \centering
    \includegraphics[width=1\textwidth]{images/moment/moment-dev-changes.png}
    \caption{Moment}
    \label{fig:moment-2.10.5-changes}
\end{figure}

\begin{figure}[H]
    \centering
    \includegraphics[width=1\textwidth]{images/moment/moment-dev-hist.png}
    \caption{Moment}
    \label{fig:moment-2.10.5-changes}
\end{figure}

\begin{figure}[H]
    \centering
    \includegraphics[width=1\textwidth]{images/moment/moment-dev-lines.png}
    \caption{Moment}
    \label{fig:moment-2.10.5-changes}
\end{figure}

\begin{figure}[H]
    \centering
    \includegraphics[width=1\textwidth]{images/moment/moment-all-changes.png}
    \caption{Moment}
    \label{fig:moment-2.10.5-changes}
\end{figure}

\subsection{Megfigyelések}

\pagebreak
\section{React}

\begin{figure}[H]
    \centering
    \includegraphics[width=1\textwidth]{images/react/react-14-15-changes.png}
    \caption{Moment}
    \label{fig:react-14-15}
\end{figure}

\begin{figure}[H]
    \centering
    \includegraphics[width=1\textwidth]{images/react/react-all-changes.png}
    \caption{Moment}
    \label{fig:react-14-15}
\end{figure}