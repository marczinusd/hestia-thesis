\chapter{Eredmények}
\label{ch:results}

\section{Vizsgált kódbázisok}

Egy kódbázisnak a következő feltételeket kell teljesíteni, hogy vizsgálni tudjuk:
\begin{enumerate}
    \item Legyen könnyen build-elhető, évekre visszamenően
    \item A tesztek legyenek könnyen futtathatóak
    \item Generálódjon coverage report a teszt futtatásokhoz és a report formátuma legyen egyike azoknak, amiket támogat a ReportGenerator\footnote{https://github.com/danielpalme/ReportGenerator} nevű .NET-es könyvtár
    \item A projekt legyen nyílt forráskódú
    \item Rendelkezzen a projekt viszonylag gazdag commit történettel -- jelen esetben az 5000+ commit és 3+ éves történet heurisztikát alkalmaztam
\end{enumerate}

Ezeket a feltételeket ugyan könnyű teljesíteni egy fejlesztő csapatnak ipari környezetben saját projektekre, sajnos nyílt forráskódú környezetben más a helyzet -- specifikusan az 1-es és 3-as pontok jelentenek problémát. A C++-ra, Python-ra és .NET Framework-re épülő projektek például egy az egyben kiesnek, mert egyik sem ad egyszerű módot arra, hogy a build-eléshez szükséges környezetet könnyen elő tudjuk állítani a projekt életciklusának akármely pontján.

Gyakorlatban a JavaScript NodeJS/NPM, .NET Core/.NET 5+ és Java ökoszisztémákra épülő projektek felelhetnek meg a korábbi feltételeknek. Ezek közül azonban a Java a szakmai hátterem miatt kiesik, a .NET Core/5+ projektek jelentős többsége pedig az 5-ös feltételt nem teljesíteni, így marad a JavaScript Node/NPM ökoszisztéma. A specifikus projekteknél külön meg fogom említeni, de általánosságban itt is leírom: még a JavaScript/NodeJS/NPM ökoszisztéma esetén is komoly problémát jelent a régi (~2015 előtti) verziók build-elése, mivel az NPM és a node, illetve a rájuk épített projektek sok változáson estek át az évek során, amit ma a projekt ismerete nélkül már nehéz reprodukálni. Ez azt jelenti, hogy a legtöbb esetben lehetetlen

Fontos leszögezni még egy dolgot. A JavaScript-es projektek elemzése egy jelentős vakfoltot fog képezni, méghozzá azért, mert a nyílt forráskódú projektek között gyakorlatilag lehetetlen olyat találni, amire őszintén azt lehet mondani, hogy rossz minőségű kódbázissal rendelkezik. Nyilván egy kódbázis minősége szubjektív, de a GitHub-on host-olt JavaScript projektek túlnyomó többsége 95\% feletti coverage-el rendelkezik. Ráadásul azok a projektek, amik 95\% alatt vannak, azok jellemzően csak a nagyon konzervatív coverage ignore path-ok miatt mutatnak ilyen értékeket -- a React projekt például 80\% körüli értéket mutatott, azonban hamar kiderült, hogy ez csak azért van, mert halott kódot tartanak a repository-ban, amit már nem hajtanak meg unit tesztek.

A fentieket figyelembe véve a következő projektekre esett a választás:
\begin{itemize}
    \item Vue: https://github.com/vuejs/vue
    \item Express: https://github.com/expressjs/express
    \item React: https://github.com/facebook/react
    \item Gatsby: https://github.com/gatsbyjs/gatsby
\end{itemize}

\section{Vue}

Elsőként a vue.js\footnote{https://github.com/vuejs/vue} kódbázisát fogjuk megvizsgálni. A Vue egy progresszív, JavaScript-alapú frontend framework. A Vue feature-ök tekintetében valahol a később taglalt Angular és React között van -- nem próbál egy kikövezett utat adni, mint az Angular, de nem csak egy specifikus szeletét fedi le a frontend fejlesztésnek, mint a React.

\lstset{language=HTML, caption={Egy egyszerű Vue komponens}}
\begin{lstlisting}
<div id="app">
    {{ message }}
</div>
\end{lstlisting}

\lstset{language=JavaScript, caption={Egy egyszerű Vue komponens JS kódja}}
\begin{lstlisting}
var app = new Vue({
    el: '#app',
    data: {
        message: 'Hello Vue!'
    }
})
\end{lstlisting}

A projekt viszonylag fiatal, fejlesztése 2016-ban kezdődött. A későbbi megfigyelések szempontjából fontos megjegyezni, hogy ugyan jelen pillanatban 338 egyedi kontribútora van a projektnek, a fejlesztés nagy része egy fejlesztőhöz, Evan You nevéhez köthető:

\lstset{caption={A vue.js top 10 kontribútora}}
\begin{lstlisting}
vue git:(dev) git shortlog -sn | head -n10
    2303  Evan You
    78  vue-bot
    47  Hanks
    34  Eduardo San Martin Morote
    32  kazuya kawaguchi
    30  chengchao
    25  katashin
    21  AchillesJ
    18  Herrington Darkholme
    15  JK
\end{lstlisting}\label{code:vue-authors}

Az analízist a vue esetében az első publikus release-től kezdjük.

\subsection{Vue 1.0}

A vue több szempontból is különös eset lesz: egyrészt az első publikus release-ig gyakorlatilag egy egyszemélyes projekt volt, másrészt pedig, ahogy azt később látni fogjuk, a Vue 2.0 egy teljes újraírása volt a projektnek.

Először vessünk egy pillantást a \ref{table:vue1-top-files} táblázatra, amiben a vue 1.0-ás változatának legtöbbet módosított fájljai láthatóak. Látható, hogy bár egy fejlesztője van a projektnek és 100\%-os coverage-el rendelkezik, már itt kialakulóban van fájloknak egy halmaza, amik vonzani fogják magukhoz a későbbi változtatásokat és javításokat. A \code{compile.js}, \code{directive.js} és \code{watcher.js} fájlokra kiemelten érdemes figyelni, mert ezek már most kiemelkednek az átlagból méret és változtatások száma tekintetében (átlag fájl méret 147, átlag változtatások száma 20 a teljes repository-ra).

\begin{table}[h]
    \centering
    \begin{tabular}{l|l|l|l|l}
        Filename      & Lifetime Authors & Lifetime Changes & Line Count & Coverage \% \\ \hline
        compile.js    & 2                & 123              & 709        & 100         \\
        directive.js  & 1                & 102              & 320        & 100         \\
        init.js       & 1                & 80               & 114        & 100         \\
        watcher.js    & 3                & 72               & 344        & 100         \\
        vue.js        & 1                & 52               & 96         & 100         \\
        lifecycle.js  & 1                & 49               & 68         & 100         \\
        data.js       & 1                & 45               & 174        & 100         \\
        lang.js       & 4                & 44               & 388        & 100         \\
        dom.js        & 2                & 42               & 362        & 100         \\
        transclude.js & 2                & 40               & 148        & 100         \\
        global.js     & 1                & 39               & 152        & 100
    \end{tabular}
    \caption{A vue 1.0-ás kiadásának leggyakrabban módosított fájljai} \label{table:vue1-top-files}
\end{table}

Érdekes továbbá a változtatások számáról készített hisztogram a \ref{fig:vue1-hist} ábrán, amiről jól látszik, hogy a 67 forrás fájlból álló vue 1.0 változtatásainak egy jelentős része 5 fájl köré csoportosul, amik méret és változtatás szám szepontjából már most az átlag többszörösét mutatják.

\begin{figure}[h]
    \centering
    \includegraphics[width=1\textwidth]{images/vue/vue1-hist.png}
    \caption{Fájlonkénti változtatások számának a hisztogramja}
    \label{fig:vue1-hist}
\end{figure}

A \ref{fig:vue1-cov-changes} ábra mutatja a teljes repository-n a fájlonkénti változtatások számát csökkenő sorrendben, illetve a másodlagos tengelyen látható a fájlonkénti coverage. A coverage egyelőre 100\% a teljes projekten. A változtatások számára illesztett görbét érdemes megjegyezni, mert visszatérő motívum lesz a többi analízis során is.

\begin{figure}[H]
    \centering
    \includegraphics[width=1\textwidth]{images/vue/vue1-lifetime-changes.png}
    \caption{A coverage és az összes változtatások számának alakulása a vue 1.0 esetében}
    \label{fig:vue1-cov-changes}
\end{figure}

A \ref{fig:vue1-changes-lines} az egyéni fájlok változtatási számát és a fájlok méretét ábrázolja. Bár van néhány fájl, mint például a \code{vue.js} és a \code{directive.js}, amik a sok változtatás ellenére kis méretűek, a kódbázis nagy részére igaz, hogy minél több módosítás tartozik egy fájlhoz, annál nagyobb -- számszerűsítve a korreláció a két adatsor között 0,59 a vue 1.0 esetében.


\begin{figure}[H]
    \centering
    \includegraphics[width=1\textwidth]{images/vue/vue1-lifetimechanges-authors.png}
    \caption{Fájlok módosítási számai és egyedi szerzőinek száma a vue 1.0 esetében}
    \label{fig:vue1-changes-authors}
\end{figure}

Végezetül a \ref{fig:vue1-changes-authors} ábra a fájlok változtatási számát és az egyedi szerzők számát mutatja. A vue ebből a szempontból különleges eset, mert ahogy a \ref{code:vue-authors} kimenetén láttuk a projekt jelentős részben egy fejlesztő kezében van és ez kiemelten igaz az 1.0-ra. A két adatsor között a vue 1.0 esetében a korreláció 0,27.

\begin{figure}[H]
    \centering
    \includegraphics[width=1\textwidth]{images/vue/vue1-lines-lifetimechanges.png}
    \caption{Fájlok módosítási számai és méretük a vue 1.0 esetében}
    \label{fig:vue1-changes-lines}
\end{figure}

\pagebreak

\subsection{Vue 2.0}

A vue 2.0 alig több, mint egy évvel a vue 1.0 után jelent meg, azonban teljes újraírása volt a projektnek, így az 1.0-ban látott adatokkal csak néhány esetben fogunk tudni párhuzamot vonni. A \ref{table:vue2-top-files} táblázat mutatja, hogy a vue 2.0 esetében melyek a legtöbbet változtatott fájlok.

Külön felhívnám a figyelmet a \code{compiler/parser/index.js} és \code{compiler/codegen/index.js} fájlokra és nem csak azért, mert ez a két legtöbbet változtatott fájl. Ezek a fájlok ismerősek lehetnek a vue 1.0-nál látott \code{compile.js} miatt, ami mint az látható a \ref{table:vue1-top-files} táblázatból az akkori állapot legtöbbet változtatott fájlja. Ugyan a git szempontjából nincs reláció a \code{compile.js}, \code{compiler/parser/index.js} és \code{compiler/codegen/index.js} fájlok között, ránézésre azonban hamar kiderül, hogy a \code{compile.js} tovább él a 2.0-ban, két külön fájlra bontva. Ez már most egy érdekes trendet mutat: hiába lett refactor-olva a \code{compile.js}, a szétbontott fájlok pontosan ugyanazt a mintát mutatják, mint az elődjük.

\begin{table}[h]
    \hspace*{-1cm}\begin{tabular}{l|l|l|l|l}
        Filename                     & Lifetime Authors & Lifetime Changes & Line Count & Coverage \% \\ \hline
        compiler/parser/index.js     & 5                & 93               & 467        & 100         \\
        compiler/codegen/index.js    & 2                & 72               & 246        & 100         \\
        render.js                    & 2                & 67               & 253        & 100         \\
        create-component.js          & 4                & 56               & 296        & 100         \\
        patch.js                     & 2                & 54               & 524        & 100         \\
        transition.js                & 1                & 47               & 270        & 100         \\
        lifecycle.js                 & 3                & 45               & 201        & 100         \\
        helpers.js                   & 2                & 34               & 145        & 100         \\
        web-runtime-with-compiler.js & 3                & 33               & 84         & 100         \\
        core/index.js                & 1                & 31               & 13         & 100
    \end{tabular}
    \caption{A vue 2.0-ás kiadásának leggyakrabban módosított fájljai} \label{table:vue2-top-files}
\end{table}

A \code{compile.js} utódait leszámítva azonban a vue 2.0 kódbázisa a \ref{fig:vue2-lifetime-changes} ábra alapján érezhetően kevésbé polarizált fájl módosítási számok tekintetében, mint a 1.0-ás kiadásban volt. Azt nehéz viszont megítélni, hogy ez fedi-e a valóságot, vagy csak a git technikalitása a teljes újraírás miatt, hiszen láttuk, hogy a \code{compile.js} két utódja teljesen tiszta lappal indult és ugyanez a minta előfordulhat kevésbé egyértelmű szituációkban is.

\begin{figure}[H]
    \centering
    \includegraphics[width=1\textwidth]{images/vue/vue2-lifetime-changes.png}
    \caption{Fájlok módosítási számai a vue 2.0.1 esetében}
    \label{fig:vue2-lifetime-changes}
\end{figure}

A módosítási számok hisztogramja a \ref{fig:vue2-hist} nagyon hasonló képet fest a korábbiakban látottakhoz. A 2.0 kódbázisának túlnyomó többsége ismét a ritkán módosított fájlokba tömörül, azonban a leggyakrabban módosított felső 10\% újfent az átlagos módosítási szám sokszorosát produkálja.

\begin{figure}[H]
    \centering
    \includegraphics[width=1\textwidth]{images/vue/vue2-hist.png}
    \caption{Fájlok módosítási számának hisztogramja a vue 2.0.1 esetében}
    \label{fig:vue2-hist}
\end{figure}

Végül vegyük szemre a \ref{fig:vue2-lines-changes} ábrát, ami a fájlok módosítási számait és méretét mutatja. Ellentétben az 1.0-ban látottakkal, a 2.0 kódbázisa esetében a fájlok méretéből rajzolt görbe sokkal közelebb követi le a módosítási számokból rajzolt görbét.

\begin{figure}[H]
    \centering
    \includegraphics[width=1\textwidth]{images/vue/vue2-lines-lifetimechanges.png}
    \caption{A vékony kliens által generált grafikonok}
    \label{fig:vue2-lines-changes}
\end{figure}

\subsection{Trendek a vue 2.0, 2.4 és a legfrissebb verzió között}

Mielőtt levonnánk néhány tanulságot, nézzük meg, hogy hogyan alakult a vue kódbázisa a 2016-ban megjelent 2.0-ás verzió és a jelenlegi, 2020-as éles verzió között. A \ref{tab:vue-changes-comp} táblázat mutatja a fájlok módosítási számát 2.0.1 és a legfrissebb verzió között, a 2.0.1-es változtatási szám szerinti csökkenő sorrendben. Ugyan ez a táblázat csak a top 10 fájlt mutatja, de itt is egyértelműen látszódik egy trend, ami egyébként a teljes, 90 fájlból álló kódbázisra is igaz: ha egy fájlt sokat kellett módosítani a projekt életciklusának elején, akkor sokat kellett módosítani később is, amíg a keveset változtatott fájlok jellemzően kevés változtatásra szorulnak a teljes életciklusuk alatt. Nem meglepő módon így a korreláció a 2.0.1-es és legfrissebb verzió vátlzotatási számai között nagyon magas, 0,965.

\begin{table}[h]
    \centering
    \begin{tabular}{l|l|l|l|l}
        Filename                  & 2.0.1 & 2.4.0 & master & delta \\ \hline
        compiler/parser/index.js  & 93    & 149   & 206    & 113   \\
        compiler/codegen/index.js & 72    & 121   & 160    & 88    \\
        render.js                 & 67    & 105   & 122    & 55    \\
        create-component.js       & 56    & 91    & 112    & 56    \\
        patch.js                  & 54    & 92    & 119    & 65    \\
        transition.js             & 47    & 68    & 73     & 26    \\
        lifecycle.js              & 45    & 76    & 96     & 51    \\
        core/index.js             & 31    & 47    & 48     & 17    \\
        render.js                 & 30    & 60    & 72     & 42    \\
        transition.js             & 30    & 49    & 54     & 24
    \end{tabular}
    \caption{A vue.js legtöbbet változtatott fájljai 2.0 és az aktuális verzió között}
    \label{tab:vue-changes-comp}
\end{table}

A \ref{fig:vue-all-files-lifetime-changes} ábra új rendezési szempont szerint mutatja be a projekt összes fájlját módosítások száma szerint, a legújabb verzióban megfigyelt érték szerint csökkenő sorrendben. Ezen az ábrán két nagyon szembetűnő tény figyelhető meg:
\begin{itemize}
    \item A fájlok módosítási számai közötti erőviszonyok négy évnyi változtatás során sem változtak, a polarizáltság változatlanul nagyon magas
    \item Új változási gócpontot képező fájlok nem alakultak ki
\end{itemize}

A \ref{fig:vue-all-delta-hist} hisztogram a vue 2.0 és a legfrissebb verzió között megfigyelt fájl módosítások deltáit csoportosítja, azaz azt, hogy a két verzió között hány változtatás érintette az adott fájlokat. Ahogy várható, ez a hisztogram tökéletesen leköveti a 2.0-tól megfigyelt változtatási számokból képzett hisztogramokat, tovább erősítve az eddig megfigyelt trendeket.

\begin{figure}[H]
    \centering
    \includegraphics[width=1\textwidth]{images/vue/vue-all-lifetime-changes.png}
    \caption{A vue legtöbbet módosított fájljainak alakulása 2016 és 2020 között}
    \label{fig:vue-all-top-files}
\end{figure}

\begin{figure}[H]
    \centering
    \includegraphics[width=1\textwidth]{images/vue/vue-all-lifetime-changes-comp-full.png}
    \caption{A vue projekt összes fájljának módosítási számai}
    \label{fig:vue-all-files-lifetime-changes}
\end{figure}

\begin{figure}[H]
    \centering
    \includegraphics[width=1\textwidth]{images/vue/vue-all-lifetime-changes-delta-hist.png}
    \caption{A vue projekt 2.0-ás és legfrissebb verzióiban lévő fájlok módosítási számainak deltái}
    \label{fig:vue-all-delta-hist}
\end{figure}

\begin{figure}[H]
    \centering
    \includegraphics[width=1\textwidth]{images/vue/vue-all-line-count.png}
    \caption{A vékony kliens által generált grafikonok}
    \label{fig:vue-all-line-count}
\end{figure}

Végül tekintsük a \ref{fig:vue-all-delta-hist} ábrát, amin a fájlok coverage-e látható. Kijelenthetjük, hogy a vue esetében a teszt coverage semmilyen moderáló hatást nem gyakorolt a változtatási gócpontokra, illetve a gócpontok növekedése nem hatott negatívan a unit tesztekre.

\begin{figure}[H]
    \centering
    \includegraphics[width=1\textwidth]{images/vue/vue-all-coverage.png}
    \caption{A vékony kliens által generált grafikonok}
    \label{fig:vue-all-coverage}
\end{figure}

\begin{table}[h]
    \centering
    \begin{tabular}{l|l|l|l}
                                          & v2.0.0 & v2.4.0 & dev  \\ \hline
        Total no of commits               & 1147   & 1692   & 2082 \\
        Sum of commits top 10 involved in & 525    & 858    & 1062
    \end{tabular}
    \caption{}\label{tab:my-table}
\end{table}

\subsection{Megfigyelések}

A vue projekt elemzése összességében sok érdekes trendet tárt fel. Láttuk például, hogy a projekt születésétől kezdve gócpontot alkotó \code{compile.js} még egy teljes refactor után is ugyanazokat a mintákat mutatta, mint korábban.

Megfigyeltük továbbá, hogy a projekt elején fennálló módosítási trendek szinte változatlanok maradnak a projekt élettartama során. Ebbe beleértendő az is, hogy új gócpontok nem igazán tudnak kialakulni, a komplexitás lokalitása változatlan marad.

Végül láttuk azt is, hogy a coverage-nek semmilyen moderáló hatása nincs ezekre a negatív trendekre, hiszen a vue projekt gócpontjai kivétel nélkül 100\%-os coverage-el rendelkeznek. Igaz viszont az is, hogy a gócpontok növekedése nem vezetett a coverage romlásához sem.

\section{Moment.js}


\begin{figure}[H]
    \centering
    \includegraphics[width=1\textwidth]{images/moment/moment-2.10.5-changes.png}
    \caption{Moment}
    \label{fig:moment-2.10.5-changes}
\end{figure}

\begin{figure}[H]
    \centering
    \includegraphics[width=1\textwidth]{images/moment/moment-2.10.5-hist.png}
    \caption{Moment}
    \label{fig:moment-2.10.5-changes}
\end{figure}

\begin{figure}[H]
    \centering
    \includegraphics[width=1\textwidth]{images/moment/moment-2.20.5-changes.png}
    \caption{Moment}
    \label{fig:moment-2.10.5-changes}
\end{figure}

\begin{figure}[H]
    \centering
    \includegraphics[width=1\textwidth]{images/moment/moment-2.20.5-auth.png}
    \caption{Moment}
    \label{fig:moment-2.10.5-changes}
\end{figure}

\begin{figure}[H]
    \centering
    \includegraphics[width=1\textwidth]{images/moment/moment-2.20.5-hist.png}
    \caption{Moment}
    \label{fig:moment-2.10.5-changes}
\end{figure}


\begin{figure}[H]
    \centering
    \includegraphics[width=1\textwidth]{images/moment/moment-dev-changes.png}
    \caption{Moment}
    \label{fig:moment-2.10.5-changes}
\end{figure}

\begin{figure}[H]
    \centering
    \includegraphics[width=1\textwidth]{images/moment/moment-dev-hist.png}
    \caption{Moment}
    \label{fig:moment-2.10.5-changes}
\end{figure}

\begin{figure}[H]
    \centering
    \includegraphics[width=1\textwidth]{images/moment/moment-dev-lines.png}
    \caption{Moment}
    \label{fig:moment-2.10.5-changes}
\end{figure}

\begin{figure}[H]
    \centering
    \includegraphics[width=1\textwidth]{images/moment/moment-all-changes.png}
    \caption{Moment}
    \label{fig:moment-2.10.5-changes}
\end{figure}

\subsection{Megfigyelések}

\section{React}

\begin{figure}[H]
    \centering
    \includegraphics[width=1\textwidth]{images/react/react-14-15-changes.png}
    \caption{Moment}
    \label{fig:react-14-15}
\end{figure}

\begin{figure}[H]
    \centering
    \includegraphics[width=1\textwidth]{images/react/react-all-changes.png}
    \caption{Moment}
    \label{fig:react-14-15}
\end{figure}