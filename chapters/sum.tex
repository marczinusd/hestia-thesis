\chapter{Összegzés}
\label{ch:sum}

\section{Eredmények}

A cél ebben a diplomamunkában az volt, hogy néhány nyílt forráskódú projekteken végezzünk el git statisztikai és coverage vizsgálatot és nézzük meg ezeket egyrészt külön-külön, másrészt olyan szempontból is, hogy milyen kapcsolatban állnak egymással.

A Vue és a Moment projekteken végzett analízis jó példája volt annak, amire \textit{Khohm et al.}\cite{khomh} is rámutatott: a kódbázisokban képesek kialakulni módosítási gócpontok, amik a projekt élettartama alatt konzisztensen ezt a mintát fogják mutatni. Láttuk azt is, amit \textit{M. Tufano et al.}\cite{codeSmells} állított, miszerint az ilyen módosítási gócpontok a közhiedelemmel ellentétben nem a kódbázis evolúciója során jönnek létre, hanem általában már az alapkövek lefektetésekor jelen vannak.
A Vue jó példa volt arra is, hogy önmagában egy gócpont szétbontása nem feltétlenül oldja meg a problémát, hiszen a \code{compile.js} egy teljes refactor után a következő kiadásban két külön gócpontként tért vissza.

Megvizsgáltuk továbbá a coverage és a módosítási gócpontok közötti relációt is. Ugyan három projekt analízise alapján ezt nem lehet konklúzívan kijelenteni, de a megfigyelések szerint a coverage, ahogy nem kódminőségi metrika, úgy nem moderáló faktor a módosítási gócpontok kialakulásában. A vue esetében látható volt, hogy a leggyakrabban módosított fájlok coverage-e 100\% volt a projekt teljes életciklusa során, hiába növekedtek verzióról verzióra. Egy kicsit bíztatóbb metrika volt azonban az, hogy a coverage a látottak alapján nem csökken ezeken a gócpontokon a növekedésük során.

A React kódbázisán láttunk egy olyan jelenséget is, ami különösen érdekes volt. A központi rendering pipeline, ami a React legnagyobb gócpontját képezte teljesen le lett cserélve az egyik kiadásban, de gyakorlatilag az első pillanattól kezdve ugyanazokat a mintákat mutatta az új renderer, mint a korábbi, hiába volt teljesen új kód.

Milyen tanulságokat vonhatunk le ezekből az eredményekből?

Az első az, hogy a git és úgy általánosságban a verziókezelők statisztikáinak analízise valóban egy jó eszköz, ha objektíven meg szeretnénk vizsgálni egy kódbázis állapotát. Ugyan az, hogy felismerünk egy fájlt, mint módosítási gócpont, az nem garantálja, hogy az a fájl code smell, vagy code smell-eket tartalmaz, de gyanakvásra adhat okot. Továbbá saját projekten alkalmazva ezt a vizsgálatot tovább lehet vinni, hiszen az analízis alkalmazható egy fájlra a sorok szintjén is, ami mélyebb problémákra is képes rávilágítani.

A második tanulság számomra az, hogy a coverage valóban nem alkalmazható minőségi metrikaként, hiszen semmilyen korrelációt nem mutatott sem maga a coverage, sem annak a változása a gócpontok egyik metrikájával sem.

\section{További kutatási lehetőségek}

Az analízisikhez szükséges adatok előkészítésénél világossá vált, hogy ugyan a program kiválóan működik, amikor egy specifikus verzió snapshot-ját készítjük el, a mélyebb analízishez több snapshot elkészítésére van szükség, amiket utána a fájlok nevei, elérési útjai és git történetük alapján össze kell kapcsolni. Ezeket az összekapcsolásokat, ahogy én is tettem, Excel-ben is el lehet végezni, azonban ez a megoldás egyrészt nem skálázódik túl jól, másrészt a git történet alapján való összekapcsolás így gyakorlatilag lehetetlen. Továbbá ha a program képes lenne összekapcsolni snapshot-okat, akkor az így generált adatok adatbányászati szempontból nagyon érdekesek lennének.

A program fejlesztésén túl érdekes lenne továbbá a prezentált analízis használata olyan ipari projekteken, amiken egyrészt generálható histórikus coverage adat, másrészt pedig nem feltétlen JavaScript-ben lettek fejlesztve, hiszen a platform és nyelv választása valamilyen szinten torzítja az itt látott analíziseket.
